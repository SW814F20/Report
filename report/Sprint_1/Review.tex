\section{Review of Sprint}\label{sec:review-sprint-anne}
This section will conclude the first sprint, and highlight how successful this sprint has been.
Furthermore, in Essence, we need to review our configuration, which can be seen in~\autoref{sec:first-configuration-table}, as we need to use the updated configuration table as input for the next sprint.

The goals for this sprint is:

\begin{itemize}
    \item Create the first configuration table 
    \item Create login system in the application
    \item Create screen overview and functionalities
    \item Create an overview board.
    \item Implement the needed functionalities in the database.
\end{itemize}.

In~\autoref{sec:first-configuration-table} we created the first configuration table, and therefore we have completed the first goal, namely, the \textit{Create the first configuration table}.

~\autoref{} highlights the process in creating the application, and how we have implemented the needed functionalities, such as the \textit{Login system, Screen overview, and the Overview Board}.

The last thing is to implement the needed functionalities in the database, which can be seen in~\autoref{}.
Here we went over what features we needed to implement, and how we did it, and in the end, we completed the last goal, namely the \textit{Implement the needed functionalities in the database}.

From all of these sections, we have completed all of the goals that we have for this sprint, and therefore this sprint has been a success.

The last thing to do is, to review the configuration table.
We have made several changes to the configuration as we have gathered more information about how we should create one, but also as we have learned more about the domain.
We have made changes to the following cells:

\begin{itemize}
    \item Leverage
    \begin{itemize}
        \item Removed JavaScript Frameworks, because we are using Flutter to create the mobile application.
    \end{itemize}
    \item Qualification
    \begin{itemize}
        \item Changed both the Qualifier and Rebuttal, to reflect the new point of view
    \end{itemize}
    \item Criteria for Architecture Expectations
    \begin{itemize}
        \item Added more expectations to the architecture, as we wanted to have a more modularized architecture in the code.
    \end{itemize}
    \item Scenarios \& Features
    \begin{itemize}
        \item Added more scenarios and more features, because of new information about the features that should be in this solution.
    \end{itemize}
    \item Value Propositions
    \begin{itemize}
        \item Added more points to Value Propositions to reflect the changes made in Scenarios and Features
    \end{itemize}
    \item Criteria for Value Propositions
    \begin{itemize}
        \item Added more criteria to reflect the new value propositions 
    \end{itemize}
\end{itemize}

The revised configuration table can be seen in~\autoref{tab:bente-configuration-table}.

\begin{landscape}
    \begin{table}[]
        \tiny
    \begin{tabular}{|l|l|l|l|l|}
    \hline
    View & \multicolumn{1}{c|} {Paradigm} & \multicolumn{1}{c|} {Product} & \multicolumn{1}{c|}{Project} & Process \\ \hline
    Value & \multicolumn{1}{c|}{Reflection} & \multicolumn{1}{c|}{Transaction} & \multicolumn{1}{c|}{Reasoning} & Appreciation \\ \hline
    Rationale & \begin{tabular}[c]{@{}l@{}}Problematic:\\ \\ Challenge: Improve the communication \\ between customers and developers in \\ an Agile Software Development Process.\\ \\ Problem: The communication between \\ customers and developers are one of the \\ most challenging aspects of the \\ development process.\end{tabular} & \begin{tabular}[c]{@{}l@{}}Leverage:\\
        \begin{minipage} [t] {0.325\textwidth} 
            \begin{itemize}
            \item Docker
            \item Flutter
            \item DotNet
            \item Entity Framework
            \item PostgreSQL
           \end{itemize} 
          \end{minipage} 
    \end{tabular} & \begin{tabular}[c]{@{}l@{}}Resolution:\\ \\ Prospect: The communication \\ between developers and \\ product owners will become easy.\\ \\ Warrant: Because software \\ development is expensive, \\ and could reduce the \\ cost and increase the quality\\ \\ Backing: Inexpensive, \\ popular platform. Extensible\end{tabular} 
            & \begin{tabular}[c]{@{}l@{}}Criteria for resolution expectations:\\ 
                \begin{minipage} [t] {0.3\textwidth} 
                    \begin{itemize}
                    \item Is the problem still worth solving
                    \item Does the proposed solution solve the problem
                    \item Have we used the correct leverage points
                   \end{itemize} 
                  \end{minipage}    
                \\ \\ Findings:\\
                \begin{minipage} [t] {0.3\textwidth} 
                    \begin{itemize}
                    \item \textit{Possible need an adapter between formalized requirements and SUT}
                   \end{itemize} 
                  \end{minipage}   
            \end{tabular} \\ \hline
    Strategy  & \begin{tabular}[c]{@{}l@{}}Elements \& Ecology:\\ 
        \begin{minipage} [t] {0.325\textwidth} 
            \begin{itemize}
            \item The product owners problem domain will be visible
            \item The product owner will have more direct communication with the developers
            \item Making the development process transparent
           \end{itemize} 
          \end{minipage} 
    \end{tabular} & \begin{tabular}[c]{@{}l@{}}Architecture:\\ 
        \begin{minipage} [t] {0.325\textwidth} 
            \begin{itemize}
            \item Digital app editor module
            \item Digital issue tracking module
            \item Digital communication module
           \end{itemize} 
          \end{minipage}\end{tabular} & \begin{tabular}[c]{@{}l@{}}Qualification:\\ \\ Qualifier: Requires understanding of the\\ formalized methodology of requirements \\ \\ Rebuttal: The app provides an \\intuitive structured interface for\\ using the methodology,\\ so learning should be easy.\end{tabular} 
            & \begin{tabular}[c]{@{}l@{}}Criteria for architecture expectations: \\
            \begin{minipage} [t] {0.3\textwidth} 
                \begin{itemize}
                \item Is the architecture feasible
                \item Is the product user-friendly
                \item Is the architecture modularized in order to support future expansions?
               \end{itemize} 
              \end{minipage} \\
            \\ Findings:\\
            \begin{minipage} [t] {0.3\textwidth} 
                \begin{itemize}
                \item \textit{Encapsulate app behaviour in flutter widgets for ease of use and ease of customization}
               \end{itemize} 
              \end{minipage}   
\end{tabular} \\ \hline
    Tactics   & \begin{tabular}[c]{@{}l@{}}Scenarios:\\ 
        \begin{minipage} [t] {0.325\textwidth} 
            \begin{itemize}
            \item Product owner used the system to create a interaction requirement.
            \item Product owner used the system to create expected behaviors of interactions.
            \item Developers can take tasks and solve them
            \item The Product owner can at any time see how a task is going and see the progress
           \end{itemize} 
          \end{minipage} 
         \end{tabular} & \begin{tabular}[c]{@{}l@{}}Features:\\ 
            \begin{minipage} [t] {0.325\textwidth} 
                \begin{itemize}
                \item App editor where the Product owner can make their interaction requirements thereby creating an interaction model.
                \item App editor where product owner can formalize expected behavior requirements for an interaction model.
                \item The app editor can suggest behavior requirements for specific interaction elements.
                \item Translate App editor to developer requirements
               \end{itemize} 
              \end{minipage} 
    \end{tabular} & \begin{tabular}[c]{@{}l@{}}Value Propositions:\\
        \begin{minipage} [t] {0.3\textwidth} 
            \begin{itemize}
            \item Translate customer requirements to product requirements.
            \item Help customer to specify edge-case behaviours they may not realize naturally.
            \item Leviate overhead for the customer in how to explain requirements.
            \item Supply an unified framework for the developers for understanding requirements.
            \item Creates a contract between customer and developers.
           \end{itemize} 
          \end{minipage} 
         \end{tabular} & 
    \begin{tabular}[c]{@{}l@{}}Criteria for value Propositions \\ expectations: \\
        \begin{minipage} [t] {0.3\textwidth} 
            \begin{itemize}
            \item Is the translation effective
            \item Does the app gives edge-cases behaviours to the customer 
            \item Does the app makes it easier to explain requirements
           \end{itemize} 
          \end{minipage} 
         \\ \\ Findings: \\
         \begin{minipage} [t] {0.3\textwidth} 
            \begin{itemize}
            \item \textit{Help the customer and developers to get started.}
           \end{itemize} 
          \end{minipage}  
        \end{tabular} \\ \hline
    \end{tabular}
    \caption{Configuration table, named Bente.}
    \label{tab:bente-configuration-table}
    \end{table}
\end{landscape}