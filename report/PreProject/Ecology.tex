\section{Ecology}\label{sec:ecology-interface}
This section will cover the process in finding the Ecology Interfaces, which we might use in the project, and rank them.

The viewpoint of Ecology is in terms of external resources.
These categories are external services, external artifacts, external repositories, and external people.
After we have gathered all of the Ecology Interfaces, we should filter them, meaning that we need to use some sort of metrics to rank them, to get an overview of which points might be better suited for us, than others.
This should not be interpreted as, we do not use them or do not care about them, but more as we know they are there, and they might be useful, however, not at the time.

Mentioned before we have four categories we need to examine, which are all externals.
The \textbf{external service} is services that we can use, without modification, which in the end, will make our solution easier to realize, as we need to do less work.\\
An example of this could be a mobile platform, which has a trivial interface, as the operating systems, mainly Android and iOS, have made it easy to utilize the mobile's features.
The \textbf{external artifacts}, could be equipment, sensors or actuators which we can access via a trivial interface, meaning that we should not do much work, to get the data from a sensor, because of the interface.
An example of this could be a Smartwatch.
It has many sensors that can be accessed via an interface.
The \textbf{external repositories} could be for collecting and storing information, that can be accessed via a trivial interface.\\
A potential repository could be GitHub, which can store and collect data, which can later be accessed via some API's, provided by GitHub 
The last thing we should look at is \textbf{external people}.
These people are some that might be interfaced with by the system.
An example of this could be a group of people, that can help you, via a chat system.
These people can be:

\begin{itemize}
    \item Specific persons
    \item Categories of people
    \item Stakeholders
    \item Authorities
    \item Social or Professional networks
    \item Organizations
\end{itemize}

\subsection{Evaluation of Ecology Interfaces}

Now we need to find potential Ecology Interfaces to each of the four categories.
In our process, we look at one category at a time and then brainstormed to get ideas of what should be there.
When we got an idea, we would talk about it, and why it should be there.

For the external services, we have these potential Ecology Interfaces:

\begin{itemize}
    \item Mobile Platform
    \item Email Platform
    \item Web Platform
    \item Board
    \item Calendar
    \item Chat 
\end{itemize}

For the external artifacts, we have these potential Ecology Interfaces

\begin{itemize}
    \item Mobile
    \item Database
    \item Smartwatch
\end{itemize}

For the external repositories, we have these potential Ecology Interfaces

\begin{itemize}
    \item GitHub
\end{itemize}

For the external people, we have these potential Ecology Interfaces

\begin{itemize}
    \item Mediator
    \item Customer
    \item Developer
    \item Agile coach
\end{itemize}

After we got all of these potential Ecology Interfaces, we need to filter them.
This can be done in different ways.
We can use \textbf{Single Idea Evaluation} and \textbf{Comparative Idea Evaluation}.
The single idea evaluation is a method to evaluate a single idea, and comparative is where we take ideas that are alternatives to each other and see what idea seems to be the best.
We will use both of these evaluation techniques.
We use the single idea evaluation to get a better understanding of the idea, and then we will use comparative idea evaluation, to rank the different ideas against each other.

\subsubsection{Single Idea Evaluation}
To evaluate a single idea, we need to use a method.
The method we will use is \textbf{SWOT}, which stands for Strengths, Weaknesses, Opportunities, and Threats.
Table \ref{tab:swot-mobile-platform-ecology} and table \ref{tab:swot-GitHub-ecology} shows the SWOT tables for Mobile Platform and GitHub, while the other SWOT analysis tables can be seen in appendix \ref{app:swot-ecology}

\begin{table}[h]
    \begin{tabular}{|l|l|}
    \hline
    Strengths & Weaknesses \\ \hline
    \begin{tabular}[c]{@{}l@{}}Widely available\\ Low location restriction\\ Familiar to the user\\ Can work on “the go”\\ Scalable\end{tabular} & \begin{tabular}[c]{@{}l@{}}Date connectivity\\ Battery powered\\ Small interfaces\\ Many platforms (OS and hardware)\\ Entry cost\end{tabular} \\ \hline
    Opportunities & Threats \\ \hline
    \begin{tabular}[c]{@{}l@{}}Automations of tasks\\ Adaptable for other scenarios\\ Multiple simultaneous users\end{tabular} & \begin{tabular}[c]{@{}l@{}}Privacy/security\\ Third party rights\\ Theft of device, potential leaks\end{tabular} \\ \hline
    \end{tabular}
    \caption{SWOT table for Mobile Platform in external services.}
    \label{tab:swot-mobile-platform-ecology}
\end{table}

\begin{table}[h]
    \begin{tabular}{|l|l|}
    \hline
    Strengths & Weaknesses \\ \hline
    \begin{tabular}[c]{@{}l@{}}Well documented API\\ Easy to work with\\ Common interface\\ No entry cost\\ Highlight progress\end{tabular} & \begin{tabular}[c]{@{}l@{}}Requires internet\\ Learning curve for the customer\end{tabular} \\ \hline
    Opportunities & Threats \\ \hline
    \begin{tabular}[c]{@{}l@{}}More transparency for the customer\\ Closer collaboration between the customer and developers\end{tabular} & Third party integration problems \\ \hline
    \end{tabular}
    \caption{SWOT table for GitHub in external repositories.}
    \label{tab:swot-GitHub-ecology}
\end{table}

\subsubsection{Comparative Idea Evaluation}
Mentioned before, we will use this evaluation technique, to rank the different ideas.
Here we will use the SWOT method, however, we will have some metrics for the four categories in SWOT.
For strengths and opportunities, the value can be ranging from 0 and 5.
For weaknesses and threats, the value can be ranging from minus 5 to 0.
The table \ref{tab:swot-ranking-external-services} shows our ranking for external repositories, as we think that all of these ideas, try to solve the same problem, and are therefore alternatives to the solution.

\begin{table}[h]
    \begin{tabular}{|l|c|c|c|c|c|}
    \hline
    Ecology Ideas & \multicolumn{2}{c|}{Strengths/Weaknesses} & \multicolumn{2}{c|}{Opportunities/Threats} & \multicolumn{1}{l|}{Overall} \\ \hline
    Mobile Platform & 5 & -1 & 3 & -1 & 6 \\ \hline
    Email Platform & 4 & -2 & 0 & 0 & 2 \\ \hline
    Web Platform & 5 & -1 & 2 & -1 & 5 \\ \hline
    Board & 4 & -3 & 2 & -2 & 1 \\ \hline
    Calendar & 3 & -1 & 0 & -1 & 1 \\ \hline
    Chat & 4 & -3 & 2 & -1 & 2 \\ \hline
    \end{tabular}
    \caption{SWOT table with values for external services.}
    \label{tab:swot-ranking-external-services}
\end{table}

The rest of the tables can be seen in appendix \ref{app:swot-ecology}

\subsection{Conclusion of Ecology}
After we have ranked all of the points, we will use these Ecology Interfaces, in further development.
These will be used when we create our configuration table, in later parts of the report.
Below is the prioritized list of the Ecology Interfaces we will look at.
The number specifies how important the Ecology Interface is:

\begin{enumerate}
    \item Mobile Platform
    \item GitHub
    \item Web Platform
    \item Mobile
    \item Database
\end{enumerate}
