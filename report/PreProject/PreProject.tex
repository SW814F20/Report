For this project, we, the project group, wanted to try a new work method, which was lectured about in the course, "Software Innovation".
The method is called \textbf{Essence}, and it is more a tool to use before a project starts, and also have some artifacts that will be used throughout the project.
We will try to follow the principles as best as we can, however, this is our interpretation of it, so there might be some errors, and so forth.
In the following sections, will we go over the Structure of Essence, and this is how we will describe them, and they might be slightly misunderstood, and they might not.
It should be mentioned that this is not a full work framework, however, it is merely a tool you use in the pre-project, and use something that is called a \textbf{configuration table}, however, we will highlight this later in the report.

So why do we want to use it?
The main reason why, is that it seems to be a good tool to use, to get a better start at a project, while we still can utilize an Agile process, such as Scrum and XP.
Furthermore, it also gives us a new view of how the working flow in a project can be.

\subsection{Structure in Essence?}
In Essence, we have \textbf{Pre-project}, which is the main focus of Essence, in our opinion.
The pre-project is as the name state before a project starts. 
The overall goal is to identify some key technologies and other important information we might need to solve the problem at hand.
The way we do this is that we have four steps we need to get through.
These are:

\begin{itemize}
    \item Find a \textbf{Challenge}
    \item Find the \textbf{Ecology Interfaces}
    \item Find the \textbf{Leverage Points}
    \item Find the \textbf{Initial Problem}, from the Challenge, Ecology Interfaces and Leverage Points
\end{itemize}

First, we need to find a challenge.
A challenge is a problem we want to solve, however, we do not know how we might solve it, what technologies we need to use, what platforms we should use and so forth.
You can think of it as an Initial problem statement.
It is narrowed down a bit, however, not enough to be a problem, that we should invest a lot of time and resources in.

After we have identified a challenge, we need to find the Ecology Interfaces and Leverage Points.
These two processes are similar, however, the focus is on different aspects.
Ecology focuses more on the externals aspects of the project, meaning that if we can use a piece of software another company has made, and they have a trivial interface, then it will make our workload lower, than if we needed to make it ourselves.
Leverage Points focuses more on the \textbf{internal} aspects, meaning that what technologies can we use to create this program that solves the problem.
These points have a crafted interface, meaning that we need to write the code ourselves.
These two processes can be made independently of each other, so if you find the Leverage Points first, and then Ecology Interfaces or vice versa, does not matter.

Now we need to create the Initial problem. 
This is done from the information gathered from the Ecology Interfaces and Leverage Points, while also looking at the challenge.
This will, in the end, make a Problem statement, and the pre-project phase ends.
Now we need to develop a solution, and here the work process, can be an Agile process, such as Scrum and XP, however, before we start, we need to create a configuration table, which is a table, that gives an overview of the focus of the project. Here the initial problem will be, and we can see what artifacts we will use, and also see why our solution will help to solve the problem.
We will go into further details when we create our first configuration table.

