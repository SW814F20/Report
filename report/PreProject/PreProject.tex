For this project, we wanted to try a new work method, which was lectured about in the course, "Software Innovation".
The process is called \textbf{Essence}.
Essence is not a full work framework, such as Scrum or Radical Unified Process, but is used together with these.
So you will typically use Essence before, to get a better start and understanding of the problem, and then go over to Scrum, where you will have Sprints. 
Essence then has a \textbf{configuration table}, which we will highlight later in the report, but the idea is, when a sprint is done, then you will review this table, and if there are major changes to it, then we will need to change focus.
This is where the Agile process helps, as it should be easy to change the focus, whereas it can be harder in Plan-driven processes, however, in our opinion you could use Essence in these processes as well. 
We will try to follow the principles as best as we can, however, this is our interpretation of it, so there might be some misunderstandings.
In the following sections, we will go over the Structure of Essence.

Why do we want to use it?
The main reason why, is that it seems to be a good tool to use, to get a better start at a project, while we still utilise an Agile process, such as Scrum and eXtreme Programming (XP).
Furthermore, it gives a good insight into factors such as; are the project on the right track and are we developing the right solution.
The overview can be easier to get, and you do not have to piece it all together, as all the information you need, is in the configuration table.

\subsection{Structure in Essence}
In Essence, the first thing to do is to start the \textbf{Pre-project}.
The pre-project is as the name state before a project starts, which will help to get a project started under uncertainty, while also identifying some key technologies and other important information we might need to solve the problem at hand.
The way we do this is that we have four steps we need to get through.
These are:

\begin{itemize}
    \item Find a \textbf{Challenge}
    \item Find the \textbf{Ecology Interfaces}
    \item Find the \textbf{Leverage Points}
    \item Find the \textbf{Initial Problem}, from the Challenge, Ecology Interfaces and Leverage Points
\end{itemize}

In Essence, we start with a challenge, which is a problem in the given problem domain.
We have some ideas about how to solve it, but we do not know everything.
This is where Ecology and Leverage will help us, to get a better understanding of the problem and how we can solve it.

After identifying the challenge, we need to find the Ecology Interfaces and Leverage Points.
These two processes are similar, however, the focus is on different aspects.
Ecology focuses more on the \textbf{external} aspects of the project, meaning that if we can use a piece of software another company has made, and they have a trivial interface, then it will make our workload smaller, than if we needed to make it ourselves.
Leverage Points focuses more on the \textbf{internal} aspects, meaning which technologies can be utilised to create this solution.
These points have a crafted interface, meaning that we need to write the code ourselves.
These two processes can be made independently of each other, so if you find the Leverage Points first, and then Ecology Interfaces or vice versa, does not matter.

Lastly, we need to create the initial problem. 
This is done from the information gathered from the Ecology Interfaces and Leverage Points, while also looking at the challenge.
In the end, this will make a problem statement, and the pre-project phase ends.
Now we need to develop a solution, and here the work process, can be an Agile process, such as Scrum and XP, however, before we start, we need to create a configuration table, which is a table, that gives an overview of the focus of the project.
Here the initial problem will be clarified and we can see which artefacts will be useful, while also seeing why our solution will help to solve the problem.
We will go into further details when we create our first configuration table.
