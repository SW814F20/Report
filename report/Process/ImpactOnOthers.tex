\section{Impact on others}
As has happened to our group, have also happened to many other companies, universities, etc. throughout the world.
The likelihood of another global pandemic, is very hard to tell, but what the current pandemic have shown us, is that the ability to adapt and work under unusual circumstances is key.
Where most software project being conducted using agile development methods require access to an onsite customer or Product Owner, the current rules of social distancing and the current shutdown of whole counties, the accessability to such an onsite customer will be limited.
Since the Product Owner is not a developer, they should not be expected to learn and use GitHub to communicate with the developers them self and thus a tool like the product of this project would come in handy.

While others may not use the same development process and workflow as we have, the tools we have utilized is still very applicable and may come in handy for others.
While different groups/teams, that being students or employees, will often have different preferences even within the same faculty/company, each group/team will have to find the tools that enable them to perform the most.

During future pandemics, the world will have a better idea of how to deal with the situation, as it has happened before.
While many things will be repeated, we see that our tool will be very useful for continuing software development, both during a pandemic but also during normal operations.

While the current pandemic may have been very annoying during this project, it has in its sense forced us to think and work as we intend the tool to be used.
Forcing the project group to work disjoint has resulted in better understanding of the problem itself.
By always assuming that no one is physically located close to one another, each team member should be able to complete their work just as effectively from remote locations.
