This chapter will talk about what our Minimum Viable Product (MVP) is, and we will in the following chapters go though how we have developed this MVP.

\section{What is our MVP?}
As we have limited time in this project, to create a product, we will focus on creating the MVP.
Our MVP will give value to both Product Owners and developers.
IT will give value to the PO, as they can formalize their requirements in their domain, and can see what the progress is.
The value for the developers is that they will have tests, they can run, to see if their solution lives up to the requirement set by the PO, and there will be no confusion about the requirements because we will handle how to generate the tests. 

The purpose of this MVP is to give the PO an idea of how to formalize their requirements, in a mobile app, meaning that they can use it anytime they want, and anywhere, as long as they have a connection to the internet.
Furthermore, it will also give the developers an idea of how the workflow will be like, and how they can get more or less instant feedback, regarding if they have solved the task at hand, or not. 

The app will be basic, however, it will showcase how to use it, and what the progress is.

Below is a ranking of the critical features we need in the MVP:

\begin{enumerate}
    \item PO should be able to create tasks and screens in the application
    \item Developers should be able to run the automated tests
\end{enumerate}

The less important features we have, which is somewhat trivial to make is also ranked according to their importance:

\begin{enumerate}
    \item PO edit screens
    \item PO can see the status of each task, via a Kanban board in the application
    \item PO can create a new application, resulting in a new GitHub repository
    \item Have a login system
\end{enumerate}

In the next couple of sprints, we will have two projects in one, meaning that we will use the time to figure out how we should make the testing module, and the other project, will be how we should create the mobile application itself.
The mobile application should be able to convert the inputs the PO puts in, into something that the testing module will understand, and can create the tests out from this data structure we will have.