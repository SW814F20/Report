This chapter will talk about what our Minimum Viable Product (MVP) is, and we will in the following chapters go though how we have developed this MVP.

\section{What is our MVP?}
A MVP is a piece of product, that gives some sort of value, with the least amount of work went into it.
It will not have many advanced features, but will just showcase the basic functionalities of the product, so the product can be tested, and futher developemnt can be based upon the test, to get all of the wanted features.
This will give the customer an idea of how good the product will be, and may also give some early signs to the success of the product. 
If the product fails, not much time is lost, because of the amount of time working on it. \cite{whatIsMVP}

With this in mind, our MVP will have the following features:

\begin{itemize}
    \item Create a new task
    \item Create a new screen
    \item Create a new application
    \item Be able to log in and see all of the user projects
    \item Edit a screen 
    \item The app should communicate with GitHub, meaning create a new repository and create new issues, which the developers will understand how to handle
    \item Have a testing module, that can automatically create tests, which the developers can run, to see if they have solved the issue or not
    \item The customer should be able to see the status of each issue, via a Kanban board in the application
\end{itemize}

All of these features will give this project a good amount of value, as it highlights how we want the customer to make requirements for the developers, and how the developers can find out, rather fast, if they have lived up to the requirements set by the customer on a given task.
Furthermore, the customer can at any time, see the progress, in terms of which tasks are not started, being worked on, and which ones are done.
 