The implications of the Covid-19 pandemic has meant that we are limited in our ability to reach any concrete measured conclusion, regarding the proposed solution.
Therefore we will try to make our conclusion on the basics of our tacit knowledge from developing software, both as students of Aalborg University, but also as business owners and employees in software development teams.

It is important for us to disclaim that all conclusions will be based on our needs and understanding, therefore they might not hold if one were to conduct real usability tests and cross-check our conclusion.
We uphold the right to reevaluate our conclusion if usability tests are ever to be conducted.

Our solutions have some obvious limitations at this stage.
The solution does not offer many UI-elements and it does not offer any interaction or behaviour-based specifications.
This renders the solution unusable for industry use, but we have managed to build an architecture that is highly modulized and expandable, therefore we find it trivial to include more UI-elements and manageable to introduce interaction and behaviour specifications.

The Design by Contract methodology showed very usefully in this solution.
We believe that our solution embodies the Design by Contract methodology, but lacks very important details.
At the current stage of the solution, there is no real method for defining pre-conditions and invariants, which is obvious violence of the Design by Contract method.
We also believe that our solutions have been too hard coupled to the Flutter environment, and that is a problem for general use. 
We, therefore, conclude that we did not manage to implement a generalized interface between the "What You See Is What You Get"-editor and the Test Generator.
We should have to build a more formal interface that was Design by Contract oriented instead of Flutter orientated.

We believe that we have built a solution that can ease the communication, highlighted in the study by Hoda, Noble, and Marshall \cite{Hoda2011TheIO}. 
Our solution allows the Product Owner to create specifications from their domain and thereby allowing them to create specifications "in the moment" that they realize the need for the functionality.
Furthermore, we believe that our solution is beneficial for the developers since it can automatically convert specifications into executable tests.
This will allow the developers to assert whether their implementation lives up to the expectations of the Product Owner within seconds.