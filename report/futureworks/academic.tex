\section{Academic}

From an academic standpoint, we believe that our project poses some quite interesting questions and opportunities.
In this section, we will suggest some of them, to us, more interesting parts.

\subsection{Research the effect of our purposed solution}
In a world where outsourcing and remote work is more and more widespread, there are new challenges that process' such as Scrum or XP do not take into account.
This is shown in the study by Hoda, Noble, and Marshall \cite{Hoda2011TheIO}, but how would the situation be if a tool like the one suggested in this report, existed?
Would it lead to a completely new set of challenges currently unknown to the industry, or would it help process'?
What would the impact be of having such a tool versus using the tools that are available today?
We see value in conducting a study inspired by the work of Hoda, Noble, and Marshall that would try to answer these questions.

\subsection{Formulating a method for the intermediate format}
Between the Customer Application and the Test Generator there is a JSON-api, and we believe it would be of value to the computer science field to try and formulate a method, that could generalize this intermediate JSON-api.
We suggest a formal method that can allow for expressing interaction on a GUI, data entities for a GUI, behaviors of a GUI, etc.
This formal method should then be able to be used as both the structure behind a "What You See Is What You Get"-environment and a Test Generator.
This suggestion is based on how Robin Milner's Calculus of Communicating System offers many beneficial properties for visualization and verification of concurrent processes, a method for expressing GUI interaction that would allow for some of the same properties would be beneficial.
If said method existed maybe it even was possible to not only be limited to traditional testing (read acceptance-, integration-, unit testing) but possible also to utilize model checking on the GUI.
Verify properties such as, there is no trace through the GUI that will lead to a loading-icon present while actually not loading.
Or there is no trace of interaction through the GUI that would make the user stuck and having to close and reopen the GUI.
Maybe it would even be possible to do equivalence checking of two GUIs.
If one were to consider this suggestion we would recommend the book "Reactive Systems - Modeling, Specification and verification" by Luca Aceto, Anna Ingólfsdóttir, Kim Guldstrand Larsen and Jiri Srba as a starting point for understanding some of the tools currently used in other areas of software systems.
