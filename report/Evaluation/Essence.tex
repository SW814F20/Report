\section{Essence}\label{sec:eval_essence}
In this section, we will reflect on Essence, and talk about some of the things that went well-using Essence, and some things that could be better, or changed for a better understanding of it.

\subsection{Ecology}
We had some problems understanding what External services and External artifacts are, what they could be, and what the difference between them is.
This lead to some discussions that could have been avoided if we had a better understanding of it, and the lines between them where clearer.

An idea we had, to this work process, is that before you classify the different potential Ecology Interfaces, you should just create a list with all of the potential Ecology Interfaces that could be used in this project.
After the creation of the list, then classify the different potential Ecology Interfaces into Ecology terms.
Because we had the problem with External services and artifacts, we thought that these two categories could be merged into one, which still captures the meaning of these two.

Regarding the SWOT method, it gave meaning to use this method in Ecology. 
It is important to know if we are using the right people or services.
One thing that helped us, was that we had written down a small piece of text to each element, in regards to SWOT, and this made it easy for us to score them, because we could, rather fast, see what the pros and cons were in a given element.
If we had not done this, this process might have been more challenging, because we needed to remember what was the pros and cons, and we needed to be on the same page, in terms of our understanding of the element in question.

Overall we liked the process in Ecology, and it was fun to give the different elements points, in regards to the SWOT method. 
Here we had some good discussions on why we thought a given element, had the strength it had and so forth.

The key thing to take away is that the lines between the different External elements, needs to be more explicit defined, and that External artifacts and services could be merged together.

\subsection{Leverage}
This process follows the same principles as Ecology, therefore many of the things said before, also applies here.
Overall it was easy to find the internal technologies, however, we had a hard time finding points to the other internal categories.

In our internal discussions, we had problems identifying what the difference between the categories in Ecology and Leverage are.
Some categories were easy to identify, while others were difficult to separate.
One solution to this is to be more specific when talking about the different categories, both in External and Internal and have more in-depth examples, which can help the reader to understand the terms better.

The reason why we say this, is that there were some assets, that we could classify both in Leverage and Ecology.
An example of this could be the GitHub API.
We have not made the GitHub API, which is why it should be an Ecology Interface, as it is trivial to create the interface, as we should just hit a certain endpoint, however, we could argue that the GitHub API, could also be in the Leverage, which means that we should make a crafted interface, so we can interact with the API.
There is some setup in the API, which means we need to write some code before we can use it, which can be argued that we then have made a crafted interface for the GitHub API.
Because of this, it was sometimes hard to classify it, because where should the GitHub API be at? 
Leverage or Ecology?
So we need a clear line in, what is a "crafted interface" and what is not.
We know that this is one extreme, but one that should be argued about and found a solution to.  

In regards to the evaluation of the different Leverage Points, we opted to use the PCRT method, to rank and describe each point.
The method itself was easy to use and work with, however, one problem we had, was that it was hard to distinguish all points, in terms of, if it is about the technology in regards to the project, or if it is only about the technology.
We might have ranked the Leverage Points differently if we only looked at the technology itself and not looked at it in the context of the project.
Therefore it could be helpful if there was a clear recommendation on how to adjust for project context or limitations.

Just as in Ecology, we propose that before starting to classify the potential Leverage points, then create a list of these Leverage points, and then classify them.

After some discussions, we think that it does not matter a lot where a given asset is classified.
If it is in Ecology or Leverage, it does not matter that much, unless you need to put in a lot of work, to get the interface to work, then it might be preferred to use the asset as a Leverage Point, then an Ecology Interface.

Our key critique regarding Leverage is that there is too much ambiguity, and it would be beneficial to try and eliminate this. 
If possible a methodical approach to determining whether a point, either External or Internal, is an Ecology or Leverage would be beneficial.
Also, guidance on how to rank the Leverage Points using the PCRT method would help.

\subsection{Configuration table}
Many of the cells in the configuration table, were easy to write, as we had made them before, e.g. put our Leverage points into the table, and so forth.
However, there where some cells, which were more problematic to write.
Not all of the cells had a good description of what should be in it, and how the focus should be.
We had some problems understanding Elements \& Ecology, and many of the cells in Tactics.

An idea of how to counter this is to have some examples of what an object is in Elements \& Ecology.
So the idea of how to think about what an object is, in general, could be explained better. 
Regarding the Tactics, we misunderstood how they should be made, and how one should formulate it.
Many in the project group, had different ways of understanding it, and because of this uncertainty, we could not find a way to write them "correctly", in terms of how it should be understood.

So the key point to take away here is that we wanted to have a more clear idea of what should be in each cell, and maybe an example of how one should write it.

\subsection{Review of sprint}
This process was straightforward because we had a "Step by Step" understanding of how we should use it.
Because we did not have a "real" customer, it was hard to simulate what their outside view is, to the process we had, however, we had one of the group members to try to have the outside view and be the customer.
This worked to a certain degree, but we can imagine how it will work.

Because of this, we did not have any major problems or reflections on this process.
