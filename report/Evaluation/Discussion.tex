\section{Discussion}

Throughout this section, we will evaluate the solution to the challenge that this project aimed to solve.

\subsection{Solution to The Challenge}

The Challenge that this project aimed to solve was as described in~\autoref{subsec:theChallenge}:

\begin{quote}
    Improve the communication between customers and developers in an Agile Software Development Process.
\end{quote}

The solution developed during this project, which is described from~\autoref{cha:sprintOne} to~\autoref{cha:sprint_4} was supposed to improve the communication between customers and developers.
However, during the project period, the world was hit by the Corona Virus and to verify that the developed solution actually solves this problem, the product should be tested on development teams and customers.

Since the project group, was both acting as customer and developer, the group itself is a valid organization to validate the solution.
Since the project group was working using the Essence working process, which is an agile process, the group would easily be able to incorporate the tool within their own project work.
Since, such a tool, is rather essential for the project work, the tool would need to be within a certain level of completeness.
The expected outcome of this project is that of a Minimal Viable Product (MVP).

The product, itself, is rather easy for the project group to implement as a part of their further development for the project itself.
Currently not all screens within the application will be supported by the application, since they make use of UI-element not yet supported by the Test Generator.
However, as the MVP suggest, the amount of supported elements in the final product should of course be a super set of the once implemented for early validation.

The main part of The challenge was to provide a tool that would improve the communication between customers and developers, which in itself would be a rather large study to validate in depth.
As developers and Product Owners we, the project group, is of the believe that the solution would be a very helpful tool for both.

Comparing our solution to others, such as Cucumber and Fitnesse, there are some key differences.
Both Cucumber and Fitnesse work by defining behaviour in plain text, however, still using their representative syntaxes.
As these two alternatives use text-based descriptions of behaviour, our system does instead use the graphical interface of the mobile application to allow for true "drag 'n drop" interaction.
Where the other solutions allow for input validation, being from Fitnesse's Table based approach, which the customer is supposed to fill, perhaps with some help from the developers, effectively making it into a truth table.
These truth tables are then compared with the evaluated values from the system and is showed in the Wiki User Interface that Fitnesse uses.
Meaning that each time the developer will have to leave their working environment to check if the behaviour is still as expected.
With our solution the developer simply needs to generate some new tests, if they have changed, and import those tests into the software project.
From the developers own development environment the state of all tests will become visible for the developers, just as the developers are used to from their Unit Testing.

Where Fitnesse is a rather old system, from 2005 \cite{FitnesseDownload}, and it has not seen mass adoption by developers, while Cucumber is a rather new alternative, from 2018 \cite{CucumberWiki}, has seen some adoption, however, mostly to large companies where they have in-house development teams developing their products.
We see this as an effect of their Product Owners being in close relations to their developers allowing for the Product Owners to either learn the syntax or create it in collaboration with the developers.
Our tool is expected to be so intuitive to use that the Product Owners should be able to use it by themselves.
