\section{Design by Contract}

Design by Contract is a method that tries to ensure at least two very important aspects of a system, one is reliability and the other is reuseable code [cite paper].
By reliability, we refer to properties such as correctness and robustness, or in simpler terms the absence of bugs [cite paper].
The method is inspired by how a "human" contract, i.e. a business contract, works.
In the paper "Applying 'Design by Contract'" by Bertrand Meyer 1992, Meyer identifies two major properties of human contracts [cite paper]:
\begin{itemize}
	\item Each party expects some benefits from the contract and is prepared to incur some obligations to obtain them.
	\item These benefits and obligations are documented in a contract document.
\end{itemize}

Meyer also identifies two benefits of having a contract document, which is [cite paper]:
\begin{itemize}
    \item It protects the client by specifying how much should be done: The client is entitled to receive a certain result.
    \item It protects the contractor by specifying how little is acceptable: The contractor must not be liable for failing to carry out tasks of the specified scope
\end{itemize}

Another finding Meyer highlights is, "what is an obligation for one party is usually a benefit for the other" [cite paper].
To explain how Design by Contract can help us with communication we will look at the theory from two perspectives, the benefits from the customer and developer relation perspective and the benefits from the product perspective.

[MANGLER TEKNISK BESKRIVELSE AF DbC ENTEN HER ELLER SOM ET SUBSECTION]

\subsection{Customer and Developer Relation}
As mentioned in [ref motivation] it is problematic for developers to get requirements and information from the customers.
The way Design by Contract can help here is that if we imagine that every unit, by unit we mean an executable piece of code that is part of the system under development, had a contract document that specified the expected behavior of the unit.
That way the developer knows that if they just make the unit uphold the contract then they have delivered what the customer expects.
This also means that we need the customer to specify these expectations in terms of a contract, and since we know from [ref motivation] that the custoemr collaboration is trublesome we need a mechanism for this that will allow the customer to do this, with in their constraints. 
We will later on, in [ref that section], explorer mechanismens that can allow for this. 
But for now lets imagine that such a mechanism does exists and lets imagine it eliminates the customer collaboration problem.

Sop with such a contract for every unit, the developer no longer need more information from the customer since they know that if they uphold the contract then the customer is happy. 
And if the customer is not happy, the problem is not with the developer but with the specified contract.
This means that we can move the discussion away from who is wrong and into how do we specify a new contract that ensures the correct behavior.
This also implies that it would properly be a good idea if the customer are not the sole authority on the contracts, they should properbly be revised by a developer before being marked as ready for development.

\subsection{The Product Perspective}

