\section{Requirements model}

As a methodology for the customers ability to create contract that the system under development (SUD) should uphold we will look at a model for specifying requirements.
More specificly we will consider the requirements model as presented in the book "Object-Oriented Software Engineering - A Use Case Driven Approach" by Ivar Jacobson [ref to book].

We will not focus on the the other models from Jacobsons recommended development process, but the requirements model is perticular usefull for our solution and therefore we will focus on how he uses the notion of usecases and actors and how that can be adobted into our solution.

Jacobson specifies two entities in formulating requirements, \textbf{Actors} and \textbf{Usecases}.
An \textbf{actor} represents a role that a user have in relation to the system, a user might take different roles when interacting with the system, thereby an \textbf{Actor} is not linked directly with a user but is more a category of users.
A \textbf{usecase} represent something that the users should be able to do with in the system. Each \textbf{usecase} takes the perspective of the user and is formulated as a complete list of events in the interaction with the system when utilizing the functionality regarded in the usecase.

\subsection{Actors}
In more general terms a actor is something that can interact with the system. That could be a person, a machine or another software running on the same machine. 
A key destingtion to make is that a, ex. person, is not limited to be only one type of actor.
Actors should more be regarded as roles in the system, so if Britta works at a store as a clerk, she will have the actor type clerk when she is working, but in her freetime she might shop in the same store and thereby take the actor role as customer. 

In a broader sense we can consider actors to be everything communicating with the system from outside the system it self.

In our solution we can use the notion of actors to identify roles with in the SUD, this will allow the developers and customers to communicate easier since they will have a unified vocabulary when talking about who communicates with the system. 
It will also help if the SUD needs boundaries with in the system, i.e. functionalities only avaible to some actors, the customers will then be able to specify a list of actors who should have access to said functionalities, which will help the developers in ensuring correctness.
The developers might even create a access layer with in the SUD based of the actors identified in the requirements or groupings of actors.

\subsection{Usecases}
A usecase is a complete set of events that specifies all the action needed for a perticular actor to achieve a goal with in the system. 
The set of events should be considered as transaktions - meaning, first the actor does something, then the system respons, then the actor does something, then the system respons and so forth. 
A usecase is always from the perspective of a perticular actor, therefore it is important to only focus on details relevant for said actor. 
A actor will have many usecases and a there will be many usecases similiar to one or another - that could be multiple actors having similar usecases or many usecases having similar event traces. 
The act of defining usecases can help discover requirements for the system, since it forces one to think about how each actor is suppose to interact with the system.

Since the usecases are a complete trace of events for a perticular action or goal, a single usecase can then be considered as a transaction system describing a feature og functionality of the system. 
The group of all usecases then specifies all the features and functionalities of the system and how they should work. 
This allows us to consider our software as a set of state transition graphs where stimulus between the actors and the system determine the behaviour of the system.

The notion of usecases does not fit perfectly into our project, since we suspect that it would be impossible for a customer to know how to construct the event traces in a realistic manner. 
But we can get a lot of inspiration by the notion of usecases. 
Consider if the customer could specify that "As a [actor]. When I..., then I should be able to..., such that..", if we were able to facilitate such requirements in our system we already have alot of information, firstly we can reason about which actors should have access to which components.
We can reason about which functionalities of which components should be exposed to the different actors.
Furthermore the format starts to feel a bit like the "Given, when, then"-model discussed in~\autoref{sec:DbC}, a reader familiar with the agile process Scrum might also recognise that it is very similiar to the User storie template often used in Scrum.
