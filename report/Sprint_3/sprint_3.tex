This chapter will highlight what progress we have made in our third sprint.

The purpose of this sprint, is to finish the solution, so it has all of the needed features highlighted in~\autoref{cha:mvp}.
Therefore the goal of this sprint is, as follow:

\begin{itemize}
    \item Finalize the JSON format for the mobile application and the test generator
    \item Implement the GitHub API into the app
    \item Finalize the test generator so it can generate the test, with the new JSON format
    \item Update screen editor to follow the JSON format
    \item Rework the work flow in app, when user logs in to edit a new screen
\end{itemize}

\section{Progess of sprint}
During sprint three, we did not make anything new on the project, because of an unforeseen assignment all group members had to do in relations to a course.
The assignment was estimated to take three workdays (24 hours), but took way longer than that.
During the initial stages of the course, the course holder told that the assignment should be made after this project was handed in, which is why we did not plan for this assignment.
In our project group, we gave everybody three workdays to do the assignment, but after realising that the assignment would take much longer, we dedicated the whole sprint to this assignment.

This, in the end, meant that we will not get as far as we planned, but we will still have the MVP as planned, with all its features.
Furthermore, some of the more advanced features that we wanted to implement, will just be kept at an idea of how one could have done it, rather than an implementation of itself.
Because of this, the goal of this sprint will be the same as in the next sprint, and the configuration table is not changed.