\section{Review of the sprint}
To conclude this sprint, and this project, we need to review our progress.
Both in terms of the goals we had, but also if our configuration table is up to date.

The first goal we had was to finalize the JSON format, so the test generator and mobile application use the same format.
This goal is completed, as we have a unified JSON format, which encaptures the properties that we want to test within in the MVP.
Another goal was to implement this newly created JSON format, both in the test generator and mobile application.
For the mobile application, we changed the code for the screen editor, so it now outputs the correct JSON, which is then stored in the database.
Regarding the test generator, we reworked what properties it should look after in the JSON format and what type of tests it will output.
A goal was to implement the needed functionalities from the GitHub API.
This goal has been completed, as we can create a new repository, which means that the Product Owner creates a new application.
Furthermore, the Product Owner can also create new tasks, which is converted to a GitHub issue, which is what the developers will need to solve.
The Product Owner can also input how important a given task is, and can also input a more detailed description of the issue at hand if needed.
The last goal we had for this sprint, was to rework our workflow in the mobile application.
We have also completed this, which means we have a working solution, that has all of the features we want it to have.

In terms of our configuration table, we believe that it is up to date, as we have not changed our focus.
However, we believe that our solution is even more needed than when we started this project, because of the Covid-19 crisis.

The reason why, is that it is hard to utilize an on-site customer, because of all of the restrictions which we have.
Therefore our solution can help because we do not need an on-site customer to be there.
The customer can be anywhere because they only need to have a mobile device and an internet connection.
They can then input what should be on a screen, and the developers can get instant feedback, in regards to if they have solved the given task or not.
The developers do not need to wait for the customer to see it, before it can be approved, meaning that the issue has been solved, and everything is good, regarding code standard that the developers might have and such.
See more details about, why our solution might be more important than before, see in~\autoref{sec:impactOnOthers}.

Because of this, we have no changes to the configuration table.