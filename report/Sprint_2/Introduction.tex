This chapter will highlight the progress we made in our second sprint.

The purpose of this sprint is to create, what we call a \textbf{Screen}, which is a screen that the customer will create, using buttons, text and other widgets, we support.
These screens will be able to be translated into auto-generated tests, using a module we will create, that enables the developers to test the code they have written, to ensure that the functionalities the customer wants are supported in the software.
Lastly, we will focus on understanding the GitHub API, as this will enables us to create issues on GitHub, which the developers will use.

The goals for this sprint is as follow:

\begin{itemize}
    \item Create Screen functionalities, such as Screen Editor and Screen Selector
    \item Begin work on a module to auto-generate tests
    \item Research the GitHub API
\end{itemize}

We will not go into details about the implementation of screens, as it follows the same implementation style as mentioned in~\autoref{sec:sprint-one-application}, however some details about the basic implementation of the auto-generate tests module, will be highlighted.
